# Concurrent Contract Checking #

Zane Shelby  
Programming Research Laboratory  
Northeastern University  

## Abstract ##

### What ###

We propose to improve the performance of run-time contract checking by
building an implementation of software contracts that exploits the
now-common multi-core architecture. Constraints on the interface
between different components of a software system will be checked
concurrently with the evaluation of the main program, thus improving
performance. For our implementation platform we will target PLT Scheme
which features a robust contract system and is well-positioned to
benefit from such an enhancement.

1. Platform
1. PLT Scheme
1. Goal: Speed up contract checking

### How ###

* spawning a new thread
* have complex monitors evaluated there
* program says what's 'complicated' via futures
* validate by measuring speedup

********************************************************************************

## Background ##

Contracts are a powerful tool for preventing software defects. Because
they are able to exploit the full power of the programming language
they are capable of representing constraints between different
components of a system that are impossible to represent using
traditional type systems. In practice, however, developers are wary of
the run-time cost imposed by the monitoring of contracts, and limit
their use to more simple constraints. An implementation of software
contracts that performed its monitoring in parallel with the
evaluation of the main program would help realize the full potential
of software contracts by freeing the developer from the performance
considerations normally imposed by their use.

* more historical information on contracts:
** Meyer, ``Design by Contracts''
** Matthias, ``Higher Order Contracts''

* blame assignment

* uses
** debugging
** glue for typed and untyped languages
*** Sam's work on typed scheme / jacob matthews on parametricity (with Findler)

## Problem ##

* contracts are good, _but_ > complexity means > overhead
** programmers are afraid of contracts
*** reference survey
*** either they remove them or disable them
*** this makes software less reliable

* future contracts
** the idea:
*** spawn a new thread to evaluate the complex contracts
** examples:

Correctness

*** refer to previous work
*** example with the missile
**** when you have outputs / blame assignment -- future contracts must not mess with the semantics of contracts checking

Validation

Challenges

* synchronization
* selection of good candidates for concurrent execution

## Goals ##

Previous work has already explored idea of evaluating software
contracts in parallel. In particular, a great deal of progress has
been on the model of parallel contract checking by Dimoulas, Pucella
and Felleisen. We will build upon the existing body of work by taking
on the following three goals:

* theoretical foundations exist
* prototype and preliminary investigation of benefits exist

We will expand upon this work by:

### Implementation ###

* robust implementation integrated with existing PLT Compiler <-- MAIN THING
** new language level (shadow effectful code)
*** investigate how make modules in new language level work with existing code
** investigate possibly using similar techniques to those used by the debugger to instrument effectful code at runtime

* lazy task
  Dynamic, not static
** future contracts were inspired by Futures (reference)
*** there has been a lot of work in the efficient implementation of this thing, esp. lazy task creation
**** we will attempt to use this technology to improve our system

* testing
** there exists a test suite, but we will extend it
** would like to find a large application on which to validate our idea
*** web server?
*** PLT Scheme has a very active community
**** will expose this to the community to get solicit feedback

* documentation
** Talk about PLT SCheme's documentation standard -- we will follow this pattern

* licensing
** Same as PLT Scheme. 

## Personal Information ##

I am a first-year graduate student in the Programming Languages
Research lab at Northeastern University. During my time working in
industry I took advantage of the comparatively limited testing
facilities provided by the now-prevalent unit testing packages
available for mainstream imperative programming languages.

* specifics
** list projects
* research interests?
